\documentclass[11pt,a4paper]{jarticle}
%
\usepackage[dvipdfmx]{graphicx,color}
\usepackage{amsmath,amssymb}
\usepackage{bm}
\usepackage{graphicx}

\usepackage{caption}
\captionsetup[figure]{format=plain, labelformat=simple, labelsep=period, font=normalsize}
\usepackage{chngcntr}
\counterwithin{figure}{section}
\counterwithin{equation}{section}

\usepackage{ascmac}
\usepackage[top=30truemm,bottom=30truemm,left=30truemm,right=30truemm]{geometry}
\usepackage{indentfirst}
\usepackage{fancyhdr}
%
\pagestyle{fancy}
\lhead{}
\rhead{}
\cfoot{\thepage}
%

\title{巨大衝突ステージにおける衝突破壊の重要性:\\
$N$体計算$\cdot$統計的手法のハイブリッドコードの開発}


\begin{document}
\maketitle



太陽系の地球型惑星は、最終段階で火星サイズの原始惑星同士が衝突合体を繰り返し形成される。この巨大衝突ステージにおいて地球や地球-月系が形成される。
一方、太陽系外で起こる巨大衝突ステージは、衝突に伴い放出される破片によりデブリ円盤が形成され、観測されている暖かいデブリ円盤(すなわち地球形成領域のデブリ円盤)を説明することができる[1]。
\\
\indent
巨大衝突ステージに形成されるデブリ円盤について調べるためには、原始惑星の長期的軌道進化と、破壊を扱うことができる計算が必要である。
しかし衝突により放出される破片の数は$10^{35}$個以上にもなり、$N$体計算ではとても扱うことはできない。
このような多数の粒子を取り扱うには、一つ一つの粒子を取り扱うのではなく、統計力学に基づいた統計的手法が有効であるが、統計的手法では、破片が重力的に集積する際にサイズ分布が非軸対称になることや、原始惑星による軌道共鳴のような、重力相互作用の取り扱いができない。
すなわち$N$体計算と統計的手法を同時に用いると、軌道進化と破壊を同時に考慮した計算を行うことができる。
\\
\indent
そこで本研究では、$N$体計算と統計的手法を組み合わせた、衝突破壊を扱うことができるハイブリッドコードの開発を行う。
多数の破片を少数のトレーサーと呼ばれるスーパー粒子に近似することで$N$体計算のコストを抑える。
またそれぞれのトレーサーの周りに扇形領域[2]を考え、その領域に入った他のトレーサーを用いて表面数密度と平均相対速度を計算し、破壊による天体の減少[3]を取り扱う。
さらに本講演では、ハイブリッドコードにより得られる、巨大衝突ステージにおけるデブリ円盤の明るさの空間分布進化についても議論する。
\\
\\
800字\\
\\

\begin{thebibliography}{9}
 \bibitem{1} Genda, H., Kobayashi, H., \& Kokubo, E. 2015, ApJ, 810, 136
 \bibitem{2} Morishima, R. 2015, Icarus, 260, 368
 \bibitem{3} Kobayashi, H., \& Tanaka, H. 2010, Icarus, 206, 735
\end{thebibliography}


\end{document}